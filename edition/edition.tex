%%
%% invención para cuarteta de cuerda
%% juan carlos paz
%% 1961
%%
%% Edition
%%
%%
%% [acb]

\setuppapersize[A3][A3]

\setuplayout[
  topspace=60mm,
  backspace=30mm
]

\setuppagenumbering[state=stop]

\usemodule[simplefonts]
\setmainfont[cynthoproregular]
\setupbodyfont[60pt]

%% rgb scale 0 - 1 == 0 - 255
\definecolor[acbbrown][r=.671,g=.475,b=.310]
\definecolor[acbcrimson][r=.602,g=.137,b=.297]
\setupcolors[state=start]

% Coloured background for cover page
\setupbackgrounds
        [page]
        [background=color,backgroundcolor=acbbrown]

\starttext

{\color[acbcrimson]{juan carlos paz}}
\blank
invención

para cuarteto de cuerda
\blank[160mm]

\setupbodyfont[28pt]
{\color[acbcrimson]{{\backslash qgem \{quinteto globales ediciones musicales\}}}


% Preface in English

\page 

\setuplayout[
  width=middle,
  topspace=25mm,
  backspace=50mm,
  location=middle,
]

\setmainfont[mintspiritno2]
\setupbodyfont[12pt]
\setupwhitespace[small]

% Revert to plain background after colour cover.

\setupbackgrounds
       [page]
       [background=none]

\startalignment[middle]
{\bfc Preface}
\blank
{\tfb invention for string quartet
\blank
Juan Carlos Paz
\blank[2mm]
1961}
\stopalignment

\setupwhitespace[small]

\blank[8mm]

{\it Life}

Juan Carlos Paz [1898---1972] was an Argentine composer, critic,
essayist, and music theorist. Paz was noted for a diverse range of
modernist musical styles, introducing many of the new European
movements in composition into Argentina. Strongly influenced by
Schoenberg and Webern, he was notably the first to introduce twelve
tone techniques to Argentina, and made heavy use of dodecaphony from
1934 until the early 1960's, when in an interesting anti-symmetry he
was the first to reject serialism in Argentina. Although relatively
unknown in the Anglophone world Paz was a highly influential figure in
modern Argentine musical development.
\blank

{\it Edition}

This score has been prepared directly from the composer's autograph
manuscript. The work was composed by Paz in a very densely notated
two-stave reduction, with fair copy parts existing in the composer's
own hand. This edition was prepared with comparative reference to both
the reduction and the parts, as there are quite a few discrepancies
between them. Where doubts exist as to what Paz intended or
differences could not be clearly resolved, notes to this effect are
provided in the Appendix. Every effort has been made to preserve the
exact notational details used by Paz in the original manuscript parts.
\blank

{\it Piece}

The work is a string quartet of one single movement composed on quite
strict twelve tone serial principles. The tone row is given by the
composer explicitly in the manuscript, and reproduced here for
reference.
\blank
(insert lilypond snippet of tone row from ms.)
\blank
According to annotations in the manuscript, the work was composed in
1961, started on 4 June 61 and finished on 12 July 61. Hence it
represents the late period of Paz and shows that he was still using
serial techniques at this late stage.
\blank

{\it Acknowledgments}

Music engraving by:
\startlines
Andrew Bernard (Melbourne, Victoria, Australia)
Ezequiel Birman (Wilde, Buenos Aires, Argentina)
Ryan McClure (Chambersburg, Philadelphia, USA)
Alex Voice (Westminster, London, UK)
Peter Wannemacher (Lyman, Maine, USA)
\stoplines
Edition concept, coordination, and direction:
\blank
Ezequiel Birman
\blank

{\it Colophon}

The open source music engraving program Lilypond was used to prepare
the score and parts, with Git used as a version control and
collaboration tool for the globally distributed group of music
engravers who set this score. Using a text-based toolchain and
contemporary software development tools proves that this choice of
technical platform allows parallel score development and a coordinated
distributed effort that is difficult if not impossible to achieve with
alternative software applications.

The music font is Lilypond's Emmentaler, developed using Metafont, and
text markup is set in the open source font Mint Spirit No 2 from
Arkandis Digital Foundry. Typesetting for the edition done with \ConTeXt.
\blank

Andrew Bernard, April 2013

% Preface in Spanish

\page

\startalignment[middle]
{\bfc Prefacio}
\blank
{\tfb invención para cuarteto de cuerda
\blank
Juan Carlos Paz
\blank[2mm]
1961}
\stopalignment



\stoptext
